% Options for packages loaded elsewhere
% Options for packages loaded elsewhere
\PassOptionsToPackage{unicode}{hyperref}
\PassOptionsToPackage{hyphens}{url}
\PassOptionsToPackage{dvipsnames,svgnames,x11names}{xcolor}
%
\documentclass[
  letterpaper,
  DIV=11,
  numbers=noendperiod]{scrartcl}
\usepackage{xcolor}
\usepackage{amsmath,amssymb}
\setcounter{secnumdepth}{5}
\usepackage{iftex}
\ifPDFTeX
  \usepackage[T1]{fontenc}
  \usepackage[utf8]{inputenc}
  \usepackage{textcomp} % provide euro and other symbols
\else % if luatex or xetex
  \usepackage{unicode-math} % this also loads fontspec
  \defaultfontfeatures{Scale=MatchLowercase}
  \defaultfontfeatures[\rmfamily]{Ligatures=TeX,Scale=1}
\fi
\usepackage{lmodern}
\ifPDFTeX\else
  % xetex/luatex font selection
\fi
% Use upquote if available, for straight quotes in verbatim environments
\IfFileExists{upquote.sty}{\usepackage{upquote}}{}
\IfFileExists{microtype.sty}{% use microtype if available
  \usepackage[]{microtype}
  \UseMicrotypeSet[protrusion]{basicmath} % disable protrusion for tt fonts
}{}
\makeatletter
\@ifundefined{KOMAClassName}{% if non-KOMA class
  \IfFileExists{parskip.sty}{%
    \usepackage{parskip}
  }{% else
    \setlength{\parindent}{0pt}
    \setlength{\parskip}{6pt plus 2pt minus 1pt}}
}{% if KOMA class
  \KOMAoptions{parskip=half}}
\makeatother
% Make \paragraph and \subparagraph free-standing
\makeatletter
\ifx\paragraph\undefined\else
  \let\oldparagraph\paragraph
  \renewcommand{\paragraph}{
    \@ifstar
      \xxxParagraphStar
      \xxxParagraphNoStar
  }
  \newcommand{\xxxParagraphStar}[1]{\oldparagraph*{#1}\mbox{}}
  \newcommand{\xxxParagraphNoStar}[1]{\oldparagraph{#1}\mbox{}}
\fi
\ifx\subparagraph\undefined\else
  \let\oldsubparagraph\subparagraph
  \renewcommand{\subparagraph}{
    \@ifstar
      \xxxSubParagraphStar
      \xxxSubParagraphNoStar
  }
  \newcommand{\xxxSubParagraphStar}[1]{\oldsubparagraph*{#1}\mbox{}}
  \newcommand{\xxxSubParagraphNoStar}[1]{\oldsubparagraph{#1}\mbox{}}
\fi
\makeatother

\usepackage{color}
\usepackage{fancyvrb}
\newcommand{\VerbBar}{|}
\newcommand{\VERB}{\Verb[commandchars=\\\{\}]}
\DefineVerbatimEnvironment{Highlighting}{Verbatim}{commandchars=\\\{\}}
% Add ',fontsize=\small' for more characters per line
\usepackage{framed}
\definecolor{shadecolor}{RGB}{241,243,245}
\newenvironment{Shaded}{\begin{snugshade}}{\end{snugshade}}
\newcommand{\AlertTok}[1]{\textcolor[rgb]{0.68,0.00,0.00}{#1}}
\newcommand{\AnnotationTok}[1]{\textcolor[rgb]{0.37,0.37,0.37}{#1}}
\newcommand{\AttributeTok}[1]{\textcolor[rgb]{0.40,0.45,0.13}{#1}}
\newcommand{\BaseNTok}[1]{\textcolor[rgb]{0.68,0.00,0.00}{#1}}
\newcommand{\BuiltInTok}[1]{\textcolor[rgb]{0.00,0.23,0.31}{#1}}
\newcommand{\CharTok}[1]{\textcolor[rgb]{0.13,0.47,0.30}{#1}}
\newcommand{\CommentTok}[1]{\textcolor[rgb]{0.37,0.37,0.37}{#1}}
\newcommand{\CommentVarTok}[1]{\textcolor[rgb]{0.37,0.37,0.37}{\textit{#1}}}
\newcommand{\ConstantTok}[1]{\textcolor[rgb]{0.56,0.35,0.01}{#1}}
\newcommand{\ControlFlowTok}[1]{\textcolor[rgb]{0.00,0.23,0.31}{\textbf{#1}}}
\newcommand{\DataTypeTok}[1]{\textcolor[rgb]{0.68,0.00,0.00}{#1}}
\newcommand{\DecValTok}[1]{\textcolor[rgb]{0.68,0.00,0.00}{#1}}
\newcommand{\DocumentationTok}[1]{\textcolor[rgb]{0.37,0.37,0.37}{\textit{#1}}}
\newcommand{\ErrorTok}[1]{\textcolor[rgb]{0.68,0.00,0.00}{#1}}
\newcommand{\ExtensionTok}[1]{\textcolor[rgb]{0.00,0.23,0.31}{#1}}
\newcommand{\FloatTok}[1]{\textcolor[rgb]{0.68,0.00,0.00}{#1}}
\newcommand{\FunctionTok}[1]{\textcolor[rgb]{0.28,0.35,0.67}{#1}}
\newcommand{\ImportTok}[1]{\textcolor[rgb]{0.00,0.46,0.62}{#1}}
\newcommand{\InformationTok}[1]{\textcolor[rgb]{0.37,0.37,0.37}{#1}}
\newcommand{\KeywordTok}[1]{\textcolor[rgb]{0.00,0.23,0.31}{\textbf{#1}}}
\newcommand{\NormalTok}[1]{\textcolor[rgb]{0.00,0.23,0.31}{#1}}
\newcommand{\OperatorTok}[1]{\textcolor[rgb]{0.37,0.37,0.37}{#1}}
\newcommand{\OtherTok}[1]{\textcolor[rgb]{0.00,0.23,0.31}{#1}}
\newcommand{\PreprocessorTok}[1]{\textcolor[rgb]{0.68,0.00,0.00}{#1}}
\newcommand{\RegionMarkerTok}[1]{\textcolor[rgb]{0.00,0.23,0.31}{#1}}
\newcommand{\SpecialCharTok}[1]{\textcolor[rgb]{0.37,0.37,0.37}{#1}}
\newcommand{\SpecialStringTok}[1]{\textcolor[rgb]{0.13,0.47,0.30}{#1}}
\newcommand{\StringTok}[1]{\textcolor[rgb]{0.13,0.47,0.30}{#1}}
\newcommand{\VariableTok}[1]{\textcolor[rgb]{0.07,0.07,0.07}{#1}}
\newcommand{\VerbatimStringTok}[1]{\textcolor[rgb]{0.13,0.47,0.30}{#1}}
\newcommand{\WarningTok}[1]{\textcolor[rgb]{0.37,0.37,0.37}{\textit{#1}}}

\usepackage{longtable,booktabs,array}
\usepackage{calc} % for calculating minipage widths
% Correct order of tables after \paragraph or \subparagraph
\usepackage{etoolbox}
\makeatletter
\patchcmd\longtable{\par}{\if@noskipsec\mbox{}\fi\par}{}{}
\makeatother
% Allow footnotes in longtable head/foot
\IfFileExists{footnotehyper.sty}{\usepackage{footnotehyper}}{\usepackage{footnote}}
\makesavenoteenv{longtable}
\usepackage{graphicx}
\makeatletter
\newsavebox\pandoc@box
\newcommand*\pandocbounded[1]{% scales image to fit in text height/width
  \sbox\pandoc@box{#1}%
  \Gscale@div\@tempa{\textheight}{\dimexpr\ht\pandoc@box+\dp\pandoc@box\relax}%
  \Gscale@div\@tempb{\linewidth}{\wd\pandoc@box}%
  \ifdim\@tempb\p@<\@tempa\p@\let\@tempa\@tempb\fi% select the smaller of both
  \ifdim\@tempa\p@<\p@\scalebox{\@tempa}{\usebox\pandoc@box}%
  \else\usebox{\pandoc@box}%
  \fi%
}
% Set default figure placement to htbp
\def\fps@figure{htbp}
\makeatother


% definitions for citeproc citations
\NewDocumentCommand\citeproctext{}{}
\NewDocumentCommand\citeproc{mm}{%
  \begingroup\def\citeproctext{#2}\cite{#1}\endgroup}
\makeatletter
 % allow citations to break across lines
 \let\@cite@ofmt\@firstofone
 % avoid brackets around text for \cite:
 \def\@biblabel#1{}
 \def\@cite#1#2{{#1\if@tempswa , #2\fi}}
\makeatother
\newlength{\cslhangindent}
\setlength{\cslhangindent}{1.5em}
\newlength{\csllabelwidth}
\setlength{\csllabelwidth}{3em}
\newenvironment{CSLReferences}[2] % #1 hanging-indent, #2 entry-spacing
 {\begin{list}{}{%
  \setlength{\itemindent}{0pt}
  \setlength{\leftmargin}{0pt}
  \setlength{\parsep}{0pt}
  % turn on hanging indent if param 1 is 1
  \ifodd #1
   \setlength{\leftmargin}{\cslhangindent}
   \setlength{\itemindent}{-1\cslhangindent}
  \fi
  % set entry spacing
  \setlength{\itemsep}{#2\baselineskip}}}
 {\end{list}}
\usepackage{calc}
\newcommand{\CSLBlock}[1]{\hfill\break\parbox[t]{\linewidth}{\strut\ignorespaces#1\strut}}
\newcommand{\CSLLeftMargin}[1]{\parbox[t]{\csllabelwidth}{\strut#1\strut}}
\newcommand{\CSLRightInline}[1]{\parbox[t]{\linewidth - \csllabelwidth}{\strut#1\strut}}
\newcommand{\CSLIndent}[1]{\hspace{\cslhangindent}#1}



\setlength{\emergencystretch}{3em} % prevent overfull lines

\providecommand{\tightlist}{%
  \setlength{\itemsep}{0pt}\setlength{\parskip}{0pt}}



 


\KOMAoption{captions}{tableheading}
\makeatletter
\@ifpackageloaded{caption}{}{\usepackage{caption}}
\AtBeginDocument{%
\ifdefined\contentsname
  \renewcommand*\contentsname{Table of contents}
\else
  \newcommand\contentsname{Table of contents}
\fi
\ifdefined\listfigurename
  \renewcommand*\listfigurename{List of Figures}
\else
  \newcommand\listfigurename{List of Figures}
\fi
\ifdefined\listtablename
  \renewcommand*\listtablename{List of Tables}
\else
  \newcommand\listtablename{List of Tables}
\fi
\ifdefined\figurename
  \renewcommand*\figurename{Figure}
\else
  \newcommand\figurename{Figure}
\fi
\ifdefined\tablename
  \renewcommand*\tablename{Table}
\else
  \newcommand\tablename{Table}
\fi
}
\@ifpackageloaded{float}{}{\usepackage{float}}
\floatstyle{ruled}
\@ifundefined{c@chapter}{\newfloat{codelisting}{h}{lop}}{\newfloat{codelisting}{h}{lop}[chapter]}
\floatname{codelisting}{Listing}
\newcommand*\listoflistings{\listof{codelisting}{List of Listings}}
\makeatother
\makeatletter
\makeatother
\makeatletter
\@ifpackageloaded{caption}{}{\usepackage{caption}}
\@ifpackageloaded{subcaption}{}{\usepackage{subcaption}}
\makeatother
\usepackage{bookmark}
\IfFileExists{xurl.sty}{\usepackage{xurl}}{} % add URL line breaks if available
\urlstyle{same}
\hypersetup{
  pdftitle={Crazy Raisins: A Raisin Classification Adventure},
  colorlinks=true,
  linkcolor={blue},
  filecolor={Maroon},
  citecolor={Blue},
  urlcolor={Blue},
  pdfcreator={LaTeX via pandoc}}


\title{Crazy Raisins: A Raisin Classification Adventure}
\author{Yasaman Baher \and Shreya Kakachery \and Eric Wong}
\date{}
\begin{document}
\maketitle

\renewcommand*\contentsname{Table of contents}
{
\hypersetup{linkcolor=}
\setcounter{tocdepth}{3}
\tableofcontents
}

\subsection{Summary:}\label{summary}

Here, we are attempting to build a classification model using the
logistic regression algorithm which use data derived from images of
raisins to predict raisin varieties, specifically Bensi and Kecimen.
From our model, we can see that many of the size-related features like
\texttt{Area}, \texttt{Perimeter}, \texttt{MajorAxisLength}, and
\texttt{ConvexArea} are strongly related. This means that these features
are redundant, and show up more than other features. Other features like
\texttt{Eccentricity} and \texttt{Extent} give us a more unique shape
information of the raisin. We can also see from the summary statistics
table that our dataset has a wide range of variability in size,
therefore, size by itself may distinguish classes but shape metrics
could help us refine classification better. With an \texttt{accuracy}
score of \texttt{\{accuracy:.3f\}} and a weighted \texttt{F1} score of
\texttt{\{weighted\_f1:.3f\}}, we can see that our classifier performs
well, and since both values are close to one another, we can assume that
the class is balanced and not heavily skewed. With our heatmap, we can
see a strong correlation between the size feature, which is likely to
dominate the classifier's decision. This, however, could be a caveat.
Since many of our features are highly correlated, our model may be
overfitting and/or relying on redundant features. The performance of our
model was aligned with what our team expected. With the high accuracy
and F1 score, we can assume that features such as \texttt{Area},
\texttt{Perimeter}, \texttt{MajorAxisLength} are meaningful and make an
impact when distinguishing between the raisins. Our confusion matrix,
however, showed that some classes were misclassified, leading to
overlapping in physical characteristics. This makes sense as it would be
harder for the model to classify raisins that are visually similar to
one another. With our model, quality control in agriculture and food
processing could benefit tremendously as it would allow them to classify
and differentiate between raisins that are in good shape, and edible and
those that are not. With a high overall accuracy, we can assume that our
model would help these industries could save labor, and have the
classifier reliably distinguish between raisin types, reducing human
error. At the same time, the misclassifications show areas where errors
could happen, which could impact labeling, packaging, or pricing
decisions if not addressed. The results of our analysis could lead to
several future directions that could be addressed. An important question
is which features contribute the most to the classification decisions,
and whether other measurements, such as color, texture, or weight, could
improve performance. Looking at other classification models could help
us create a more reliable model, helping us yield higher accuracy or
better class-specific performance. Furthermore, we have to be mindful of
how well the classifier generalizes to new batches of raisins, as this
could reveal potential overfitting and indicate whether the model
performs reliably on unseen data.

\subsection{Introduction}\label{introduction}

Raisins are dried grapes although other small berries and fruit may be
dried using the same methodology of raisins. Just like their undried
form, raisins of different varieties may differ in taste, chewiness,
sweetness, etc. They provide a variety of health benefits which include:
``\ldots{} a better diet quality and may reduce
appetite.''(Olmo-Cunillera et al.~1) The analysis below will attempt to
predict which variety of raisins are based on data derived from multiple
images taken of 900 samples. The dataset is obtained from the UCI
database archive and contains numeric features all indicating the
properties of the sampled raisin. If our model can accurately predict
the species of raisins given it's measured properties, we can avoid
eating raisins that are not to our preference.

\begin{longtable}[]{@{}lllllllll@{}}
\toprule\noalign{}
& Area & MajorAxisLength & MinorAxisLength & Eccentricity & ConvexArea &
Extent & Perimeter & Class \\
\midrule\noalign{}
\endhead
\bottomrule\noalign{}
\endlastfoot
0 & 87524.0 & 442.246011 & 253.291155 & 0.819738 & 90546.0 & 0.758651 &
1184.040 & Kecimen \\
1 & 75166.0 & 406.690687 & 243.032436 & 0.801805 & 78789.0 & 0.684130 &
1121.786 & Kecimen \\
2 & 90856.0 & 442.267048 & 266.328318 & 0.798354 & 93717.0 & 0.637613 &
1208.575 & Kecimen \\
3 & 45928.0 & 286.540559 & 208.760042 & 0.684989 & 47336.0 & 0.699599 &
844.162 & Kecimen \\
4 & 79408.0 & 352.190770 & 290.827533 & 0.564011 & 81463.0 & 0.792772 &
1073.251 & Kecimen \\
... & ... & ... & ... & ... & ... & ... & ... & ... \\
895 & 83248.0 & 430.077308 & 247.838695 & 0.817263 & 85839.0 & 0.668793
& 1129.072 & Besni \\
896 & 87350.0 & 440.735698 & 259.293149 & 0.808629 & 90899.0 & 0.636476
& 1214.252 & Besni \\
897 & 99657.0 & 431.706981 & 298.837323 & 0.721684 & 106264.0 & 0.741099
& 1292.828 & Besni \\
898 & 93523.0 & 476.344094 & 254.176054 & 0.845739 & 97653.0 & 0.658798
& 1258.548 & Besni \\
899 & 85609.0 & 512.081774 & 215.271976 & 0.907345 & 89197.0 & 0.632020
& 1272.862 & Besni \\
\end{longtable}

\begin{table}

\caption{\label{tbl-info}High-level summary of the features in the
raisin dataset.}

\centering{

` \textless class `pandas.core.frame.DataFrame'\textgreater{}
RangeIndex: 900 entries, 0 to 899 Data columns (total 7 columns): \#
Column Non-Null Count Dtype\\
--- ------ -------------- -----\\
0 Area 900 non-null int64\\
1 MajorAxisLength 900 non-null float64 2 MinorAxisLength 900 non-null
float64 3 Eccentricity 900 non-null float64 4 ConvexArea 900 non-null
int64\\
5 Extent 900 non-null float64 6 Perimeter 900 non-null float64 dtypes:
float64(5), int64(2) memory usage: 49.3 KB

`

}

\end{table}%

\begin{longtable}[]{@{}
  >{\raggedright\arraybackslash}p{(\linewidth - 14\tabcolsep) * \real{0.0631}}
  >{\raggedleft\arraybackslash}p{(\linewidth - 14\tabcolsep) * \real{0.0901}}
  >{\raggedleft\arraybackslash}p{(\linewidth - 14\tabcolsep) * \real{0.1712}}
  >{\raggedleft\arraybackslash}p{(\linewidth - 14\tabcolsep) * \real{0.1712}}
  >{\raggedleft\arraybackslash}p{(\linewidth - 14\tabcolsep) * \real{0.1441}}
  >{\raggedleft\arraybackslash}p{(\linewidth - 14\tabcolsep) * \real{0.1261}}
  >{\raggedleft\arraybackslash}p{(\linewidth - 14\tabcolsep) * \real{0.1171}}
  >{\raggedleft\arraybackslash}p{(\linewidth - 14\tabcolsep) * \real{0.1171}}@{}}

\caption{\label{tbl-describe}Statistical summary of numerical features
in the raisin dataset.}

\tabularnewline

\toprule\noalign{}
\begin{minipage}[b]{\linewidth}\raggedright
\end{minipage} & \begin{minipage}[b]{\linewidth}\raggedleft
Area
\end{minipage} & \begin{minipage}[b]{\linewidth}\raggedleft
MajorAxisLength
\end{minipage} & \begin{minipage}[b]{\linewidth}\raggedleft
MinorAxisLength
\end{minipage} & \begin{minipage}[b]{\linewidth}\raggedleft
Eccentricity
\end{minipage} & \begin{minipage}[b]{\linewidth}\raggedleft
ConvexArea
\end{minipage} & \begin{minipage}[b]{\linewidth}\raggedleft
Extent
\end{minipage} & \begin{minipage}[b]{\linewidth}\raggedleft
Perimeter
\end{minipage} \\
\midrule\noalign{}
\endhead
\bottomrule\noalign{}
\endlastfoot
count & 900 & 900 & 900 & 900 & 900 & 900 & 900 \\
mean & 87804.1 & 430.93 & 254.488 & 0.781542 & 91186.1 & 0.699508 &
1165.91 \\
std & 39002.1 & 116.035 & 49.9889 & 0.0903184 & 40769.3 & 0.0534682 &
273.764 \\
min & 25387 & 225.63 & 143.711 & 0.34873 & 26139 & 0.379856 & 619.074 \\
25\% & 59348 & 345.443 & 219.111 & 0.741766 & 61513.2 & 0.670869 &
966.411 \\
50\% & 78902 & 407.804 & 247.848 & 0.798846 & 81651 & 0.707367 &
1119.51 \\
75\% & 105028 & 494.187 & 279.889 & 0.842571 & 108376 & 0.734991 &
1308.39 \\
max & 235047 & 997.292 & 492.275 & 0.962124 & 278217 & 0.835455 &
2697.75 \\

\end{longtable}

\begin{figure}

\centering{

\pandocbounded{\includegraphics[keepaspectratio]{"../eda_scatter_plot.png"}}

}

\caption{\label{fig-measurement-scatterplot}Scatterplot of the two major
measurements.}

\end{figure}%

Figure: Plotting the two major measurements as a scatterplot to see the
distribution of both varieties of raisins

\begin{figure}

\centering{

\pandocbounded{\includegraphics[keepaspectratio]{"../figures/eda_correlation_heatmap.png"}}

}

\caption{\label{fig-pearson-corr-matrix}Pearson correlation matrix of
each numerical feature.}

\end{figure}%

Figure: Pearson correlation matrix of each numerical feature

\section{Methods}\label{methods}

\subsection{Data}\label{data}

The dataset used in this project is the digitized raisin images by İ̇lkay
Çinar, Murat Koklu, and Sakir Tasdemir from Selcuck University (Çinar,
Koklu, and Tasdemir 2019). The data can be obtained from
{[}\href{https://archive.ics.uci.edu/dataset/850/raisin}{here}{]} and
was imported through the ucimlrepo python library. Each row represents a
measurement of a raisin belonging to either the Besni or Kecimen
variety.

\subsection{Analysis}\label{analysis}

The LogisticRegression algorithm was used to build the classification
model to predict the species of a raisin given its measured shape
properties. The data was split \texttt{r\ int((1-test\_size)*100)}\% /
\texttt{r\ int(test\_size*100)}\% into the training and test datasets
respectively. We obtained an accuracy of 87.56\% with our model.

\begin{Shaded}
\begin{Highlighting}[]
\CommentTok{\# Added conversion of y to numpy array to avoid warning when fitting model}
\NormalTok{y }\OperatorTok{=}\NormalTok{ np.array(y[}\StringTok{\textquotesingle{}Class\textquotesingle{}}\NormalTok{])}
\end{Highlighting}
\end{Shaded}

\begin{Shaded}
\begin{Highlighting}[]
\NormalTok{clf }\OperatorTok{=}\NormalTok{ LogisticRegression(max\_iter}\OperatorTok{=}\DecValTok{2000}\NormalTok{)}
\NormalTok{clf.fit(X\_train, y\_train)}
\end{Highlighting}
\end{Shaded}

\begin{verbatim}
/opt/conda/envs/dockerlock/lib/python3.9/site-packages/sklearn/utils/validation.py:1408: DataConversionWarning: A column-vector y was passed when a 1d array was expected. Please change the shape of y to (n_samples, ), for example using ravel().
  y = column_or_1d(y, warn=True)
\end{verbatim}

\begin{verbatim}
LogisticRegression(max_iter=2000)
\end{verbatim}

\begin{Shaded}
\begin{Highlighting}[]
\NormalTok{y\_pred }\OperatorTok{=}\NormalTok{ clf.predict(X\_test)}
\NormalTok{accuracy }\OperatorTok{=}\NormalTok{ accuracy\_score(y\_test, y\_pred)}
\NormalTok{accuracy}
\end{Highlighting}
\end{Shaded}

\begin{verbatim}
0.8755555555555555
\end{verbatim}

\section{Results and Discussion}\label{results-and-discussion}

We can see that the dataset is quite balanced, as shown in
\textbf{?@fig-class-distribution}.

\begin{figure}

\centering{

\pandocbounded{\includegraphics[keepaspectratio]{../figures/eda_class_distribution.png}}

}

\caption{\label{fig-logreg-conf-matrix}Class Distribution of Raisin
Variety.}

\end{figure}%

Figure: Class Distribution of Raisin Variety

The confusion matrix of our model Figure~\ref{fig-logreg-conf-matrix}
indicates that the model correctly predicts more kecimen raisins than
besni. We may improve this model with more samples, hyperparameter
optimization, and feature engineering.

\begin{figure}

\centering{

\pandocbounded{\includegraphics[keepaspectratio]{../results/models/raisin_model_confusion_matrix.png}}

}

\caption{\label{fig-logreg-conf-matrix}Confusion matrix of the
LogisticRegression model.}

\end{figure}%

Figure: Confusion matrix of the LogisticRegression model

\begin{figure}

\centering{

\pandocbounded{\includegraphics[keepaspectratio]{"../results/models/raisin_model_feature_importance.png"}}

}

\caption{\label{fig-coefs-barplot}Bar plot of Feature Coefficients.}

\end{figure}%

Figure: Bar plot of Feature Coefficients

\begin{Shaded}
\begin{Highlighting}[]
\ImportTok{from}\NormalTok{ sklearn.metrics }\ImportTok{import}\NormalTok{ f1\_score}

\NormalTok{f1 }\OperatorTok{=}\NormalTok{ f1\_score(y\_test, y\_pred, average}\OperatorTok{=}\StringTok{\textquotesingle{}weighted\textquotesingle{}}\NormalTok{)}
\NormalTok{f1}
\end{Highlighting}
\end{Shaded}

\begin{verbatim}
0.8751851851851853
\end{verbatim}

\section*{References}\label{references}
\addcontentsline{toc}{section}{References}

\phantomsection\label{refs}
\begin{CSLReferences}{1}{0}
\bibitem[\citeproctext]{ref-cinar2019raisin}
Çinar, İ̇lkay, Murat Koklu, and Sakir Tasdemir. 2019. {``Raisin
Dataset.''} \url{https://archive.ics.uci.edu/dataset/850/raisin}.

\end{CSLReferences}




\end{document}
